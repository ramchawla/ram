\documentclass[fontsize=11pt]{article}
\usepackage{amsmath}
\usepackage[utf8]{inputenc}
\usepackage[margin=0.75in]{geometry}

\usepackage{xcolor}
\usepackage{setspace}
\usepackage{hyperref}

\newcommand{\blue}[1]{\textcolor{blue}{#1}}
\newcommand{\red}[1]{\textcolor{red}{#1}}
\newcommand{\link}[1]{\href{#1}{\blue{#1}}}

\title{CSC111 Project Proposal: ``Ram'' --- A Beginner's Introductory Programming Language Using ASTs in Python}
\author{Will Assad, Zain Lakhani, Ariel Chouminov, Ramya Chawla}
\date{Tuesday, March 16, 2021}

\begin{document}

\maketitle

\section*{Problem Description and Research Question}
Modern programming languages are built using \blue{Abstract Syntax Trees (ASTs)}, a recursive data structure that represents written code \cite{david}. Programmers write code in the form of large strings, which is then parsed into ASTs in which the computer can understand and execute \cite{david, ASTs}. As humans, we often take this necessary translation for granted, given that we can simply read programs as text \cite{david}. The computer, however, interprets this text, translates it several times, and then produces an output with incredibly high speed.

Programming languages present humans the opportunity to interact with computers in unique ways. One of these languages is \blue{Python}, a relatively easy scripting language to learn. While it has many advantages as a first programming language for new programmers, there are some key disadvantages. Mainly, it differs structurally from other programming languages in terms of indented blocks and variables that represent multiple data types. This means it offers beginner programmers an introduction into the world of programming while making their transition into other languages quite difficult. 

The other problem with introducing programming to new students of Computer Science is getting them to understand what a particular syntax represents in English. The intuition behind writing certain lines of code can hide from those who lack experience. For example, there is no obvious reason why the line of Python code \texttt{print([(x, x \% 2 == 0) for x in range(10)])} should display a list of numbers from 0 to 9 and whether or not they are even. Hence, \blue{\textbf{the goal of this project is to create a programming language (Ram) that is easily translatable to English and helps better prepare beginner programmers to use other programming languages.}} That is, we're looking to provide our users with the ability to convert basic English sentences, following the syntax of our language, into lines of Ram code.

\section*{Computational Plan}

Before diving directly into the code, we will need to define all the syntactical structure of Ram code. Already mentioned is the readability of this code and how easily it can be translated to English. Below is a table containing various expressions in Python and how they would be represented in Ram.

\begin{center}
    \begin{spacing}{1.1}
    
        \begin{tabular}{|l|l|} 
             \hline
             
             \blue{Python Syntax} & \blue{Ram Syntax} \\ 
             
             \hline
             \texttt{for x in \red{range(5)}:} & \texttt{loop with x \red{from 0 to 4} \{ } \\ 
             \quad \quad \texttt{print(x)} & \quad \quad  \texttt{display x} \\
              & \} \\
             
             \hline
             
             \texttt{\red{x} = 5} & \texttt{\red{set integer} x to 5} \\
             
             \hline
             
             \texttt{\red{letters} = `hello'} & \texttt{\red{set text} letters to ``hello''} \\
             
             \hline
             
             \texttt{\red{def} f(x: int) -> None:} & \texttt{\red{new function} f takes x \{ } \\
             \quad \quad \texttt{x = x - 5} & \quad \quad \texttt{\red{reset x} to x - 5} \\
             \quad \quad \texttt{print(x)} & \quad \quad \texttt{display x} \\
              & \} \\
             
             \hline
             
             \texttt{\red{var1} = 10} & \texttt{\red{set integer} var1 to 10} \\ 
             \texttt{\red{var2} = 15 } &               \texttt{\red{set integer} var2 to 15} \\
             \texttt{print(var1 < var2)} & \texttt{display var1 less than var 2} \\

              
             \hline
        \end{tabular}
    
    \end{spacing}
    
    $ $
    
    \small{Table 1: Comparison of Python and Ram Syntax}
    
\end{center}

Our program will consist of three main parts. First, we will define various statements and expressions in Python using ASTs including but not limited to functions, for-loops, if-statements, and lists. We will then parse Ram code into the ASTs we have created in Python. Lastly, we will have an interactive component for the user, where they can write Ram code and run it via the command line.

ASTs will be created through a combination of Python's \blue{\texttt{ast} module} along with some statements and expressions defined from scratch \cite{pyAST}. We will draw partly from the CSC111 course notes \cite{david} as a starting point, but will eventually dive much deeper. Of greater difficulty will be the process of parsing.

Parsing, in programming, is the process of analyzing a string of symbols and or words which are presented in either natural language, computer language, or data structures \cite{def}. This syntactic analysis attunes to the rules of grammar in a manner which allows the computer to adequately and accurately interpret the text in order to provide the necessary output \cite{def}. As our goal is to create a structure which allows users to write lines of code following syntactic rules which resemble general grammar rules, we will use the aid of Python's \blue{\texttt{nltk} module} \cite{pyNLTK}. Through the assistance of this Natural Language Toolkit, our program will be able to analyze the text provided in English and be able to translate it accordingly in order to provide the appropriate result. 
For Ram, parsing will be somewhat simplified as Ram is fairly keyword based; the first word of a statement is always a keyword.

\begin{itemize}
    \item If a statement begins with ``set'' or ``reset'', it must be an assignment statement.
    
    \begin{itemize}
        \item A variable assignment statement is unambiguous.
        \item The type of value the variable stores is provided.
    \end{itemize}
    
    \item The keyword ``loop'' is analogous to a for-loop with range in Python.
    
    \begin{itemize}
        \item The loop-variable is always an integer.
        \item The code to be repeated is nested in \{ brackets \}.
    \end{itemize}
    
    \item The keywords ``new function'' are analogous to the keyword ``def'' in Python.
    
    \begin{itemize}
        \item Functions never have a return value.
        \item The block of the function is nested in \{ brackets \}.
    \end{itemize}
    
    \item The initial keyword will always define the rest of the statement.
    
    \begin{itemize}
        \item For most statements, this will work on a line by line basis.
        \item Code blocks in \{ brackets \} will result in subtrees of the relevant tree object.
        \item Code in the function block will be a subtree of a function AST.
    \end{itemize}
    
\end{itemize}

To use Ram, code can be edited in any text editor and saved with the \texttt{.ram} file extension. Lastly, running the code will be implemented using a command line approach:

\begin{verbatim}
    ~ % ram <filename>.ram
    SAMPLE OUTPUT FROM FILE
\end{verbatim}


\begin{thebibliography}{9}

\bibitem{david}
David Liu. \textit{CSC111 Lectures}. University of Toronto. 2021.

\bibitem{ASTs}
Dennis Howe. \textit{Free On-line Dictionary of Computing}. Internet Encyclopedia Project. 2008.

\bibitem{pyAST}
\textit{Abstract Syntax Trees}. Python Documentation. Available from: \link{https://docs.python.org/3/library/ast.html}

\bibitem{pyNLTK}
\textit{Natural Language Toolkit}. Python Documentation. Available from: \link{https://www.nltk.org/}

\bibitem{def}
Chapman, Nigel P. \textit{LR Parsing: Theory and Practice, Cambridge University Press}. 1987.

\end{thebibliography}

\end{document}